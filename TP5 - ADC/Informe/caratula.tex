%COMANDOS DE INFORMACION DE MATERIA
\newcommand{\codigoMateria}{86.07} %una "macro" para definir codigo de materia
\newcommand{\nombreMateria}{Laboratorio de Microprocesadores}
\newcommand{\turno}{Curso 1}
%INFORMACION DEL TP
\newcommand{\fechaentrega}{16/Jun/2020}
\newcommand{\nroTP}{4}
\newcommand{\descripcionTP}{Interrupción Externa}


\newcommand{\resumen}{En el presente trabajo se desarrollará el diseño de programa que consiste en detectar y manejar la funcionalidad de \textbf{interrupción de hardware} del microcontrolador \textit{Atmel MEGA 328P}. Para ello el programa deberá controlar el comportamiento de dos diodos LED a través de una entrada digital que se detectara mediante dichas interrupciones.}



\newcommand{\tituloTP}{Trabajo práctico Nº \nroTP}
\newcommand{\facultad}{Facultad de Ingeniería}
\newcommand{\universidad}{Universidad de Buenos Aires}

%COMANDOS DE DATOS PERSONALES

\newcommand{\nombrerodri}{Vazquez, Rodrigo}
\newcommand{\padronrodri}{98934}
\newcommand{\mailrodri}{rvazquez@fi.uba.ar}

%MARGENES
\marginsize{2cm}{2cm}{1cm}{1.5cm} %izquierda, derecha, arriba, abajo

%HEADERS Y FOOTERS DEL INFORME
\pagestyle{fancy} % seleccionamos un estilo
\fancyhead{}
\fancyfoot{}
\lhead{\includegraphics[width= 2.5 cm]{imagenes/logofiuba.png}} % texto izquierda de la cabecera
\rhead{ \codigoMateria\,- \nombreMateria\, - \turno} % texto centro de la cabecera
\cfoot{\thepage}

%OTRAS OPCIONES DE FORMATO
\newcommand{\HRule}{\rule{\linewidth}{1mm}} %linea negra de separacion\

\date{} %saca la fecha
\raggedbottom %evita espacios en blancos grandes entre imagenes y textos

%PORTADA
\vspace*{-12mm}%esto, no entiendo como funciona pero hace lo que quiero (dejar un pequeño espacio)




\begin{center}

\LARGE{\textsc{ \tituloTP }}\\[.3cm]
%\HRule \\[0.1cm]

\LARGE{\textbf{\descripcionTP}}\\[0.01cm]
\small{\fechaentrega}

\HRule\\[0.3cm]
\end{center}

%\centering{\textbf{\underline{Grupo N°.:}}1} \\
\begin{tabbing}
	\begin{tabular}{ l l l }
	   
	    \padronrodri & \nombrerodri & \mailrodri\\
	    
	\end{tabular}
\end{tabbing}
\textbf{\underline{Docentes:}} \\
\begin{tabbing}
	\begin{tabular}{ l }
	   
	    Campiglio, Guillermo Carlos\\
	    Stola, Gerardo Luis\\
	    Cofman, Fernando\\ 
	    Salaya Velazquez, Juan Guido
	    
	\end{tabular}
\end{tabbing}
\centering 
\HRule
\begin{abstract}
\justify
\resumen
\medskip
\end{abstract}
\justify
\HRule
\medskip

\tableofcontents
\pagebreak